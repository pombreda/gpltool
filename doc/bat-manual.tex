\documentclass[10pt]{article}
\usepackage{url}
\usepackage{a4}

\pagestyle{empty}

\setlength{\parskip}{4pt}

\author{Armijn Hemel -- Tjaldur Software Governance Solutions}
\title{Binary Analysis Tool User Manual}

\begin{document}

\maketitle
\thispagestyle{empty}

\section{Introducing the Binary Analysis Tool}

The Binary Analysis Tool is a tool that 

\section{Design of the Binary Analysis Tool}

BAT consists of several scripts written in Python.

\subsection{\texttt{bruteforce.py}}

The main program in BAT is the \texttt{bruteforce.py} tool. It takes a brute
force approach to scanning a binary and has no prior knowledge of how a binary
is constructed. Instead it finds out by applying several methods, including
searching for known identifiers ...

\subsubsection{Unpack scans}

\begin{itemize}
\item file systems: cramfs, ext2/ext3/ext4, ISO9660, JFFS2, SquashFS (specific
variants), UBIFS
\item compressed files: ar, ARJ, BASE64, BZIP2, compressed Flash, CAB, CPIO,
EXE (specific compression methods only) GZIP, InstallShield (old versions),
LZIP, LZMA, LZO, RAR, RPM, TAR, XZ, ZIP
\item media files: GIF, ICO, PNG
\end{itemize}

\subsubsection{Leaf scans}

Leaf scans, or program scans, are scans that are run on every single file
that has been unpacked, including ...

\begin{itemize}
\item fast string searches: BusyBox, dproxy, ez-ipupdate, iptables, iproute,
libusb, Linux kernel, loadlin, RedBoot, U-Boot, vsftpd, wireless-tools,
wpa-supplicant
\item dynamic library dependencies (ELF files only)
\item Linux kernel module license (Linux kernel modules only)
\end{itemize}

\subsubsection{Enabling and disabling scans}

The \texttt{bruteforce.py} tool uses a configuration file that is in Windows
INI format. There is a default configuration file that enables most of the
scans by default. To disable a scan it can be outcommented in the file (by
starting the line with the \texttt{\#} character), or by removing it from the
configuration file.

\subsubsection{Blacklisting and priorities}

In BAT blacklists are used to prevent some scans from running on a particular
byte range, because other scans have already covered these bytes, or will cover
them.

The most obvious example is the ext2 file system: in a normal setup (no
encryption) it is trivial to see the content of all the individual files when
ax ext2 file system image is opened. This is because this file system is
mostly a concatenation of the data, with some meta data associated with the
files in the file system.

If another compressed file is in the ext2 file system it could be that it will
be picked up by BAT twice: once it will be detected inside the ext2 file system
and once after the file system has been unpacked by the ext2 file system
unpacker.

Other examples are:

\begin{itemize}
\item cpio (files are concatenated with a header and a trailer)
\item tar (files are concatenated with some meta data)
\item RPM (files are in a compressed archive with some meta data)
\item ar and DEB
\item some flavours of cramfs
\item ubifs
\end{itemize}

To avoid duplicate scanning and false positives it is therefore necessary to
prevent other scans from running on the byte range already covered by one of
these files.

In BAT this is achieved by using blacklists. All unpack scans have a parameter
called \texttt{blacklist} which is consulted everytime a file is unpacked. If
a file system offset is in a blacklist the scan could use the next offset, or
skip scanning the entire file, depending on the scan.

The blacklist is set for every file individually and is initially empty. If a
scan is successful it adds a byte range to the blacklist and changes it for
..

The scans are run in a particular order to make the best use of blacklists. The
order of scans is determined by the \texttt{priority} parameter in the
configuration file. The file systems and concatenated files mentioned above
have a higher priority and are scanned earlier than other scans that could also
give a match. It is not a fool proof system, but it works well enough.

\subsubsection{Pretty printing}

\subsection{Interpreting the results}

BAT outputs its results as XML on standard output. After redirecting the output
to a file it is possible to look at this file using a commandline tool such as
\texttt{xml\_pp} or a webbrowser such as Mozilla Firefox.

In the future a graphical user interface will be wrapped around BAT that will
make it easier to interpret the results in a more interactive way.

The XML file starts with metadata, such as:

\begin{itemize}
\item date, plus time of the scan (local time of the computer, in UTC)
\item name of the file
\item SHA256 cryptographic checksum of the file, uniquely identifying it
\item size of the file
\item filetype as determined by \texttt{file} on a Linux system
\item relative path inside the unpacked system, plus the absolute path inside
the file system, which is useful for later analysis
\end{itemize}

If any of the scans were successful the results of the individual scans will
be included in the element \texttt{scans}.

For each unpacking scans the following attributes are reported:

\begin{itemize}
\item name of the scan (corresponding to the name of the scan in the
configuration file)
\item offset in the parent file of the compressed file or file system
\end{itemize}

\section{Binary Analysus Tool extratools collection}

To help with unpacking non-standard file systems, or standard file systems for
which there are no tools readily available on Fedora or Ubuntu there is also
a collection of tools that can be used by BAT to unpack more file systems.
These tools are not part of the standard distribution, but have to be installed
separately. They are governed by different license conditions than the core BAT
distribution.

Currently the collection consists of:

\begin{itemize}
\item modified version of \texttt{cramfsck} that enables unpacking cramfs
file systems
\item unmodified version of \texttt{unyaffs} that enables unpacking for some
(but not all) YAFFS2 file systems.
\item various versions of \texttt{unsquashfs} that enable unpacking variants
of SquashFS. These versions have either been lifted from vendor SDKs, the
OpenWrt project, or upstream SquashFS project.
\end{itemize}

\section{Binary Analysis Tool knowledgebase}

BAT comes with a mechanism to use a database backend. The default version of
BAT only unpacks file systems and compressed files and runs a few simple checks
on the leaf nodes of the unpacking process.

In the paper "Finding Software License Violations Through Binary Code Clone
Detection" by Hemel et. al. (ACM 978-1-4503-0574-7/11/05) presented at
the Mining Software Repositories 2011 conference a method to use a database
with strings extracted from source code was described. This functionality is
is available in the ranking module in the file \texttt{ranking.py}. This code
is not enabled by default, but it has to be explicitely enabled in the
configuration for \texttt{bruteforce.py}.

To give good results the database that is used needs to be populated with as
many packages as possible, from a cross cut of all of open source software, to
prevent bias towards certain packages: if you only would have BusyBox in your
database, everything would look like BusyBox.

To populate the database you need to download and process open source packages.
Be aware that this is a costly operation: a download of all source packages of
a snapshot of Debian (which would give you a few versions for most packages) is
about 60 GiB. Together with all of GNU, GNOME, most BusyBox versions, and a few
versions of KDE you have about 100 GiB of compressed source code archives.
Processing these files (including licensing information) will take a long time
and the resulting database will also be quite big (at least around 20 GiB).

If you don't want to spend much time on downloading packages and processing
these packages, please contact Tjaldur Software Governance Solutions for
purchasing a copy of a fully preprocessed database at \url{info@tjaldur.nl}.

\subsection{Crawling FTP mirrors}

In the directory \texttt{crawlers} a sample crawler is given for downloading
source packages from the GNU project. The code is fairly straightforward. The
script (written in Python) tries to intelligently download packages from a GNU
FTP mirror, skipping packages that are in a blacklist (either parts of the
filename, whole directories, or extensions) or that have already been
downloaded in the past.

The default configuration points to a mirror in the Netherlands. It is advised
to update the location of the GNU mirror in the configuration file. Specifying
a local mirror will result in faster downloads. It will also not put an
unnecessary strain on the FTP mirror mentioned in the default configuration
file.

\subsection{Generating the package list}

The code and license extractor wants a description file of which packages to
process. This file is hardcoded to \texttt{LIST} relative to the directory that
contains all source archives. The reason we have a specific file is that some
packages do not follow a consistent naming scheme. By using this extra file we
can cleanup names and make sure that source code archives are recognized
correctly.

The file contains four values per line:

\begin{itemize}
\item name
\item version
\item archivename
\item origin (or ``unknown'' if not specified)
\end{itemize}

separated by whitespace (spaces or tabs). An example would look like this:

\begin{verbatim}
amarok	2.3.2	amarok-2.3.2.tar.bz2	kde
\end{verbatim}

This line says that the package is \texttt{amarok}, the version number is
\texttt{2.3.2}, the filename is \texttt{amarok-2.3.2.tar.bz2} and the file
was downloaded from the KDE project.

There is a helper script (\texttt{generatelist.py}) to help generate the file.
It can be invoked as follows:

\begin{verbatim}
python generatelist.py -f /path/to/directory/with/sources
\end{verbatim}

The output is printed on standard output, so you want to redirect it to a file
called \texttt{LIST} (as expected by the string extraction script) and
optionally sorting it first:

\begin{verbatim}
python generatelist.py -f /path/to/directory/with/sources | sort >
   /path/to/directory/with/sources/LIST
\end{verbatim}


\texttt{generatelist.py} tries to determine the name of the package by
splitting the file name on the right if it encounters a \texttt{-} (dash)
character. This is not always done correctly because a package uses multiple
dashes, or because it does not contain a dash. In the latter case an error
will be printed on standard error, informing you that a file could not be
added to the list of packages and it should be added manually.

It is advised to manually inspect the file after generating it to ensure the
correctness of the package names. Packages can have been renamed for a number
of reasons:

\begin{itemize}
\item upstream projects decided to use a new name for archives (abiword
archives for example were renamed from \texttt{abi-\$VERSION.tar.gz} to
\texttt{abiword-\$VERSION.tar.gz}).
\item a distribution has renamed packages to avoid clashes during installation
and allow different versions to be installed next to eachother.
\end{itemize}

In these cases you need to change the names of the packages, otherwise
different versions of the same package will be recorded in the database as
different packages, which will confuse the rating algorithm and cause it to
give suboptimal results.


\subsection{Running the extraction program}

The program to extract strings from sourcecode is
\texttt{batchextractprogramstrings.py}. It parses the file generated by
\texttt{generatelist.py}, unpacks the files (currently only \texttt{tar.gz}
and \texttt{tar.bz2} files, plus variants are supported) and scans each
individual source code file (at the moment just C/C++ files, assembler files,
Java/Scala files and ActionScript files) for strings and, if enabled, licenses.

\texttt{batchextractprogramstrings.py} internally uses \texttt{xgettext} to
extract all strings between double quotes from source code. These are the
strings that are likely to end up in a binary, such as console messages, error
messages or messages and so on. \texttt{batchextractprogramstrings.py} simply
invokes \texttt{xgettext} and parses the results.

\begin{verbatim}
python batchextractprogramstrings.py -f
    /path/to/directory/with/files -d /path/to/database
\end{verbatim}

\subsection{License scanning}

\texttt{batchextractprogramstrings.py} has a few commandline options. The most
important is whether or not to also extract licenses from the source code files.
License extraction is done using the Ninka license scanner, with support
for the Nomos license scanner from FOSSology being added in the near future.
This option is disabled by default for a few reasons:

\begin{itemize}
\item scanning licenses adds a significant performance penalty to scanning the
code (about 1100\% when scanning uClibc 0.9.30.1), even with many
optimizations (like deduplication) in place.
\item there is currently no code in the Binary Analysis Tool that can make use
of this information.
\end{itemize}

If you want to enable license scanning, you will have to install Ninka first
and change a few hardcoded paths in \texttt{batchextractprogramstrings.py} that
point to the main Ninka script.

\subsection{Intelligently making sets of packages when using multiple machines}

One way to tackle the tackle the problem of having to process huge amounts of
source code is to use multiple machines and later merge database. By adding a
little bit of intelligence to preparing the sets of source packages you can
save a lot of processing time. Many packages have duplicate files, especially
if they are just different versions of the same package. A minor patch version
could differ as little as one file. If the extractor already has seen a file, it
will not scan it again, so it makes sense to have as many similar packages
scanned on the same machine, for example all versions of a single package. If
license scanning is enabled it makes sense to have as many packages from the
same project scanned on the same machine, since coding styles often result in
identical license statements, which the scanner already takes into account.

\subsection{Enabling the database}

Currently the database is hardcoded to \texttt{/tmp/master}. It can be changed
in \texttt{ranking.py} in the following line:

\begin{verbatim}
  conn = sqlite3.connect(os.environ.get('BAT_SQLITE_DB', '/tmp/master'))
\end{verbatim}

In the future it will be possible to set the location of the database in the
configuration file for \texttt{bruteforce.py}.

\end{document}
