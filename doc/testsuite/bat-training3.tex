\documentclass[11pt]{beamer}

\usepackage{url}
\usepackage{tikz}
%\author{Armijn Hemel}
\title{Using the Binary Analysis Tool - part 3}
\date{}

\begin{document}

\setlength{\parskip}{4pt}

\frame{\titlepage}

\frame{
\frametitle{Subjects}
In this course you will learn:

\begin{itemize}
\item scan order in the Binary Analysis Tool
\item to configure the Binary Analysis Tool
\item to browse results of a scan made with the Binary Analysis Tool
\end{itemize}
}

\frame{
\frametitle{Scan order in the Binary Analsysis Tool}

\begin{enumerate}
\item the binary is read and file type specific identifiers (from a hardcoded list) are searched for in the binary
\item prerun scans are run to tag files to filter out specific file types later
\item unpacking scans are run to unpack any compressed files or file systems and scans 1 - 3 are run recursively
\item file specific scans are run on all unpacked files
\item post run scans are run for each individual file
\end{enumerate}
}

\frame{
\frametitle{Identifier search}
The identifier search uses a hardcoded list of identifiers that indicate where a certain file starts or stops. Not all file types have identifiers, but most do.

The locations of identifiers are passed to later scans, which can use this information to work in a more efficient way.
}

\frame{
\frametitle{Prerun scans}
Prerun scans are run to determine the type of file and to tag it. Tags can be used by later scans to ignore files: a scan to unpack a file system won't work on a graphics file or a text file.

Some tags that are set by prerun scans as shipped by BAT are:

\begin{itemize}
\item text
\item binary
\item graphics
\item audio
\item elf
\end{itemize}
}

\frame{
\frametitle{Unpacking scans}
}

\frame{
\frametitle{File specific scans}
}

\frame{
\frametitle{Postrun scans}
}

\frame{
\frametitle{Configuring the Binary Analysis Tool}

The Binary Analysis Tool is highly configurable and uses plugins. These plugins can be enabled and disabled via a configuration file.

The configuration file is in Windows INI format and contains several parts:

\begin{itemize}
\item general configuration
\item configuration directives for ``prerun'' scans
\item configuration directives for ``unpack'' scans
\item configuration directives for per file scans
\end{itemize}

The general configuration is mandatory, the other directives are optional.
}

\begin{frame}[fragile]
\frametitle{General configuration directive}

\begin{verbatim}
[batconfig]
multiprocessing = no
module = bat.simpleprettyprint
output = prettyprintresxml
\end{verbatim}

\end{frame}

\frame{
\frametitle{Mandatory scan configuration options}
The configuration for each type of scan has a few mandatory options:

\begin{itemize}
\item \texttt{type} - \texttt{prerun}, \texttt{unpack}, \texttt{program} or \texttt{postrun}
\item \texttt{module} - Python module (including package) the scan can be found
\item \texttt{method} - the method for the scan
\item \texttt{enabled} - \texttt{yes} enables a scan, \texttt{no} disables a scan
\end{itemize}
}

\frame{
\frametitle{Optional scan configuration options}

\begin{itemize}
\item \texttt{priority}
\item \texttt{noscan}
\item \texttt{magic}
\item \texttt{description}
\end{itemize}
}

\begin{frame}[fragile]
\frametitle{Prerun scans configuration directive}

\begin{verbatim}
[checkXML]
type        = prerun
module      = bat.prerun
method      = searchXML
priority    = 100
description = Check XML validity
enabled     = yes
\end{verbatim}

\end{frame}

\begin{frame}[fragile]
\frametitle{Unpacking scans configuration directive}

\begin{verbatim}
[7z]
type        = unpack
module      = bat.fwunpack
method      = searchUnpack7z
priority    = 1
magic       = 7z
noscan      = text:xml:graphics:pdf:bz2:gzip:lrzip:audio:video
description = Unpack 7z compressed files
enabled     = yes
\end{verbatim}

\end{frame}

\frame{
\frametitle{Examining the results of the Binary Analysis Tool}
The output of the Binary Analysis Tool is written as a tar archive. The tar archive consists of:

\begin{itemize}
\item full directory tree of unpacked files (if any)
\item Python pickles with the results of the scan
\item (optional) pictures with results of the scan
\end{itemize}

The results can be viewed using the Binary Analysis Tool result viewer.
}

\frame{
\frametitle{Using the Binary Analysis Tool result viewer}
The Binary Analysis Tool result viewer is a Python program using wxPython. It can be invoked using the command:

\texttt{batgui}

which will launch the GUI.
}

\end{document}
