\documentclass[11pt]{beamer}

\usepackage{url}
\usepackage{tikz}
%\author{Armijn Hemel}
\title{Using the Binary Analysis Tool - part 2}
%\date{January 17, 2011}
\date{}

\begin{document}

\setlength{\parskip}{4pt}

\frame{\titlepage}

\frame{
\frametitle{Subjects}

In this course you will learn:

\begin{itemize}
\item to determine the type of a file
\item to find dependencies of dynamically linked binary files (ELF format)
\item to extract human readable strings from binary files
\item to extract function names from dynamically linked ELF files
\item how these methods are implemented in the Binary Analysis Tool
\end{itemize}
}

\frame{
\frametitle{Before we start}
It is important to keep in mind that the methods outlined below are only for finding evidence. Whether or not someone violates a license is a legal question, not a technical one. It is easy to jump to incorrect conclusions. Interpretation of legal results should be done by legal professionals, not engineers.
}

\frame{
\frametitle{Examining individual files}

Possible types of analysis that could be performed on an individual file:

\begin{itemize}
\item file type
\item dynamically linked libraries
\item extracting human readable strings
\item match strings extracted from binaries with source code
\item extracting function names from ELF executables
\item match function names extracted from binaries with source code
\end{itemize}
}

\frame{
\frametitle{Interesting files for compliance analysis}
There are various types of files that are interesting in for license compliance:

\begin{itemize}
\item executables (ELF, bFLT, Java, Windows executables)
\item libraries
\item Linux kernel images
\end{itemize}

Other files, such as audio, video or graphics files, can be interesting too for trademarks, or licensing (Creative Commons), but are outside of the scope of this course.
}

\frame{
\frametitle{Determining file types}
You can make a good first guess with the \texttt{file} command, which uses \texttt{libmagic}.

\texttt{libmagic} uses the so called ``magic database'' which contains descriptions for many sorts of files. It is usually found in \texttt{/usr/share/magic} on Linux systems.

By looking at a significant number of bytes and it can make an educated guess which type a file has, since many files have unique identifiers or characteristics.
}

\frame{
\frametitle{Excercise: determine the type of files}

Use the \texttt{file} command to determine the type of various files on your system.
}

\frame{
\frametitle{Drawbacks of using \texttt{file} and \texttt{libmagic}}
There are some drawbacks to using commands like \texttt{file} or tools that use \texttt{libmagic}:

\begin{itemize}
\item the descriptions in the magic database can and will change over time and sometimes contains errors.
\item the magic database is not complete.
\item some file types don't have a fixed magic type.
\item not the entire file is considered, but just a number of bytes (up to a few hundred), so false positives can happen, especially if extra data has been appended to the file (this happens frequently with firmware images).
\end{itemize}
}

\frame{
\frametitle{File type verification in BAT}
In BAT data from \texttt{libmagic} is used for reporting, but not for searching and unpacking. It is used sometimes as a verification step.

Instead, a more \textit{conservative} approach is used:

\begin{itemize}
\item BAT uses standard tools to verify, even though this sometimes costs more time. For example \texttt{gifinfo} is used to verify GIF files. Compared to \texttt{file} these tools often actually do take the whole file into account.
\item BAT errs on the safe side: if not sure (or sure enough) that a file is of a certain type then it is treated as an unknown file type.
\end{itemize}
}

\frame{
\frametitle{Inspecting dynamically linked ELF files}
In case the file is an ELF executable (executable program) or shared object (library for dynamic linking) it is important to find the run time dependencies (also called \textit{shared libraries}) because linking creates a \textit{derivative work} which is very significant for fullfilling licensing conditions.
}

\frame{
\frametitle{ELF dynamic linking 101}
ELF executables can be linked in two ways:

\begin{itemize}
\item static linking: all dependencies (C library, other libraries) are included into the final executable.
\item dynamic linking: dependencies are recorded at build time, but only resolved when the program is run (run time)
\end{itemize}

For license requirements it is very significant to see what a program links to (either dynamically or statically). If a program is dynamically linked you need to inspect the file to find the dependencies.
}

\begin{frame}[fragile]
\frametitle{Using \texttt{readelf} to display shared libraries}
The best tool to display shared libraries is to use the \texttt{readelf}, which is part of GNU binutils.

The command:

\texttt{readelf -d | grep NEEDED}

will display the shared libraries that are used, for example (output formatted using \texttt{cut}):

\begin{verbatim}
$ readelf -d /bin/ls | grep NEEDED | cut -f 4- -d " "
            Shared library: [libselinux.so.1]
            Shared library: [librt.so.1]
            Shared library: [libcap.so.2]
            Shared library: [libacl.so.1]
            Shared library: [libc.so.6]
\end{verbatim}
\end{frame}

\frame{
\frametitle{Why not use \texttt{ldd}?}

One reason to not use \texttt{ldd} is that \texttt{ldd} uses the dynamic linker on the \textit{local} system to find dependencies.

It might seem that an executable or library is linked with a certain library, while in fact it is not and just a dependency on the \textit{local} system.

Also, it will not work for executables compiled for different architectures.
}

\frame{
\frametitle{Excercise: find the dynamically linked libraries of a file}
\begin{enumerate}
\item Look at several dynamically linked executables on your system with \texttt{readelf}. Also look at them with \texttt{ldd} and try to spot differences.
\item repeat, but with files from the test firmware(s)
\end{enumerate}
}

\frame{
\frametitle{Extracting human readable strings from a file}
With the \texttt{strings} program it is possible to extract sequences of human readable text (in the ASCII range)
from a file (by default sequences of length 4 or greater):

\texttt{strings /path/to/file}

These sequences are (possibly) string constants that are not stripped by the compiler:

\begin{itemize}
\item output messages
\item error messages
\item debug messages
\item etcetera
\end{itemize}
}

\frame{
\frametitle{Excercise: extracting strings from a file using \texttt{strings}}

Examine a few executables on your system by running \texttt{strings} on it:

\texttt{strings /path/to/file}

It is useful to play with the size parameter (\texttt{-n}) and see how output willdiffer: smaller sequences will give more output, but also have significantly more noise. Example to extract sequences of size 6:

\texttt{strings -n 6 /path/to/file}
}

\frame{
\frametitle{Matching strings from binaries with source code}

After extracting strings you need to find out where they come from. Good sources are:

\begin{itemize}
\item search engines
\item database containing strings from source code
\end{itemize}

BAT has mechanisms available to use databases and there are companies that offer pregenerated databases.
}

\frame{
\frametitle{Unique matches}
If a string can only be found in one package (counting different versions, or rebranded packages, as one package) it is a \textit{unique match}.

If there are many unique matches, it is a good sign that a certain piece of software was used in the binary.

However: you should never base a conclusion on just one unique match.
}

\frame{
\frametitle{Non-unique matches}

There are several reasons why strings might appear in multiple programs:

\begin{itemize}
\item massive reuse of (open source) code and copying of code between programs (``cloning''): good examples are \texttt{zlib} and \texttt{libjpeg}
\item protocols and standards (POSIX, RFCs, W3C, C, etcetera)
\item coding styles (GNU, others)
\end{itemize}
}


\frame{
\frametitle{String extraction in the Binary Analysis Tool}
In the Binary Analysis Tool only strings with length of 5 or longer are considered. There are two ways to compare strings:

\begin{enumerate}
\item checks with a few hardcoded strings (that are most likely unique for some packages)
\item advanced check using database backend (needs database)
\end{enumerate}

The first method is very quick, the second method is much more accurate, but requires a lot of data. Generating a database is out of scope for this course, but scripts to generate databases are freely available.
}

\frame{
\frametitle{Extracting function names from a dynamically linked ELF file}
}

\frame{
\frametitle{Excercise: extract function names from a dynamically linked ELF file}
Extract function names from several dynamically linked ELF files on your system using the following command:

\texttt{readelf --dyn-syms /path/to/file | grep FUNC | grep -v UND | tr -s " " | cut -f 9 -d " "}
}

\frame{
\frametitle{Matching function names from binaries with source code}

After extracting function names you need to find out where they come from, just like with strings. Good sources are:

\begin{itemize}
\item search engines
\item database containing function names extracted from source code
\end{itemize}

BAT has mechanisms available to use databases and there are companies that offer pregenerated databases.
}

\frame{
\frametitle{Function name extraction in the Binary Analysis Tool}
}

\end{document}
